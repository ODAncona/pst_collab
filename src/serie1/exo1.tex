\begin{exo}
  \donnee{Un groupe de consommateurs a réalisé une étude pour analyser le service offert par 200 employés de	divers restaurants. On s’intéresse à une possible relation entre la qualité du service et la qualification du personnel (diplômé d’une école hôtelière ou non). Les résultats de l’enquête figurent dans le tableau ci-dessous :
    \begin{center}
      \begin{tabular}{lll}
        & Bon service & mauvais service \\
        \toprule
        Diplôme & 61 & 28 \\
        \midrule
        Sans diplôme & 30 & 81 \\
        \bottomrule
      \end{tabular}
    \end{center}}
  Rappel : $P(A) = \frac{\text{Nombre de cas favorables}}{\text{Nombre de cas possibles}}$
  \begin{subexo}{Calculer les probabilités d’avoir choisi une personne :}
  \begin{enumerate}[parsep=0cm, itemsep=3mm, topsep=3mm]
    \item dont le service est qualifié bon.
    \begin{enumerate}
      \item[ ] \text{Ici on réunis les bons services des personnes avec et sans diplôme. On a donc : }
      \item[ ] \begin{center}{Cas fav.} = 61 + 30 = 91\end{center}
      \item[ ] \text{Ensuite on calcul les cas possibles. On a donc : }
      \item[ ] \begin{center}{Cas tot.} = 61 + 28 + 30 + 81 = 200\end{center}
      \item[ ] \text{On applique la formule et on obtient : }
      \item[ ] \begin{center}$\dfrac{91}{200} = 0,455$\end{center}
    \end{enumerate}
    \item non diplômée.
    \begin{enumerate}
      \item[ ] \text{Ici on réunis les personnes sans diplôme, peut importe la qualité du service. On a donc : }
      \item[ ] \begin{center}{Cas fav.} = 30 + 81 = 111\end{center}
      \item[ ] \text{Ensuite on calcul les cas possibles. On a donc : }
      \item[ ] \begin{center}{Cas tot.} = 61 + 28 + 30 + 81 = 200\end{center}
      \item[ ] \text{On applique la formule et on obtient : }
      \item[ ] \begin{center}$\dfrac{111}{200} = 0,555$\end{center}
    \end{enumerate}
    \item diplômé dont le service est bon.
    \begin{enumerate}
      \item[ ] \text{Ici on réunis les personnes avec diplôme et une bonne qualité de service. On a donc : }
      \item[ ] \begin{center}{Cas fav.} = 61\end{center}
      \item[ ] \text{Ensuite on calcul les cas possibles. On a donc : }
      \item[ ] \begin{center}{Cas tot.} = 61 + 28 + 30 + 81 = 200\end{center}
      \item[ ] \text{On applique la formule et on obtient : }
      \item[ ] \begin{center}$\dfrac{61}{200} = 0,305$\end{center}
    \end{enumerate}
  \end{enumerate}
  \end{subexo}
  \begin{subexo}{Quelle hypothèse faites-vous pour déterminer ces probabilités ?}
    Toutes les probabilités sont \textbf{équiprobables}. C'est pour cela qu l'on peut utiliser la formule énnoncée dans les rappels de l'exercice.

  \end{subexo}
\end{exo}
