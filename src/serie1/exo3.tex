\begin{exo}
  \donnee{Dix personnes attendent l'ascenseur au rdc d'un immeuble formé de 6 étages (sans le rez-de-chaussée). La probabilité qu'une personne quitte l'ascenseur à l'un des six étages est la m
  même pour tous les étages. Observateurs de l'expérience, nous supposons que les personnes sorent au hasard de
  l'ascenseur indépendamment les une des autres; une fois sorties, elles n'y rentrent plus.}
  \begin{subexo} {Déterminer la probabilité qu’aucune personne sortira de l’ascenseur au 5ième étage.}
    La probabilité peut être vu comme $\frac{\text{nombre de cas favorables (1)}}{\text{nombre de cas totaux (2)}}$

      \begin{enumerate}[label=(\arabic*)]
        \item Les cas favorables sont les cas où les personnes descendent à un autre étage que le 5ième.
        Autrement dit combien existe-il de possibilité pour 10 personnes de sortir sur les 5 autres étages ?
        Pour la première personne, elle a 5 choix (les 5 étages qui restent.) la seconde a aussi 5 choix etc etc.
        On a donc $5^{10}$ cas favorables
        \item les cas totaux sont simplement $6^{10}$ puisque on autorise à sortir à l'étage 5
      \end{enumerate}
      On a donc $$a) \left(\frac{5}{6}\right)^{10}$$

  \end{subexo}
  \begin{subexo}{Déduire de a) la probabilité que l'ascenseur s'arretera au 5ième étage}
      On nous demande ici de compter la probabilité qu'on s'arrête au 5ième étage. C'est à dire qu'une personne s'y arrete ou que 2 personnes s'y arretent ou que 3 personnes etc etc.
    Il est plus simple de faire l'inverse : 1 - la probabilité que personne ne s'arrete au 5ième. Et cette dernière probabilité est justement celle que nous avons calculer en (a)
    $$b)1 - \left( \dfrac{5}{6} \right)^{10}$$
    \end{subexo}
  \begin{subexo}{Une autre manière de voir l'exo a)}
    Nous avons l'équi-probabilité: $ P({\text{sortir à l'étage i}}) = \frac{1}{6}$
    \begin{enumerate}
      \item 			Soit $A_1$ l'évenement la personne 1 sort à l'étage 5. ($P(A_1) = \frac{1}{6}$)
\item 			Soit $A_k$ l'évenement la personne $k, k\in 1..10$ sort à l'étage 5.

    \end{enumerate}
    Nous avons $\overline{A_k}$ l'évenement "la personne $k$ ne sorte PAS à l'étage 5" $P(\overline{A_k}) = 1 -\frac{1}{6}$
    Pour que personne ne sortent à l'étage 5 il faut que la personne 1 ne sorte pas à l'étage 5 ET que la personne 2 ne sorte pas etc.
    \newline
    Puisque les évènements sont indépendents, on a $$P(A \cap B) =P(A) \cdot{ P(B)} $$
    donc $$ P\left( \bigcap_{k=1}^{10}\overline{A_{k}}\right) = \left(\frac{5}{6}\right)^{10}$$
  \end{subexo}
\end{exo}
