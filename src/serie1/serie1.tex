\documentclass[11pt,a4paper]{article}

%
% Packages
%
\usepackage[T1]{fontenc} % Output font encoding for international characters
\usepackage[utf8]{inputenc} % Required for inputting international characters
\usepackage{amsmath,amsthm,amssymb}
\usepackage[svgnames]{xcolor}
\usepackage[english,french]{babel}
\usepackage{multicol}
\usepackage{pstricks,pst-node,pst-text,pst-poly,pst-3d}
\usepackage{enumitem}
\usepackage{sidenotes}
\usepackage{graphicx} % Required to insert images
\usepackage{array}
\usepackage{booktabs} % Required for better horizontal rules in tables
\usepackage{titlesec} % Required for modifying sections

%
% Constant and Variables
%
\setlength{\textwidth}{175mm}
\setlength{\textheight}{255mm}
\setlength{\oddsidemargin}{-10mm}
\setlength{\topmargin}{-15mm}
\setlength{\parskip}{0.2cm}

\definecolor{vertfonce}{rgb}{0,0.5,0}

%
% Custom commands
%

\newcommand{\ds}{\displaystyle}
\newcommand{\scr}{\scriptscriptstyle}
\newcommand{\bs}[1]{\ensuremath{\boldsymbol{#1}}}

%
% Header
%
\newcommand{\header}[2]{

\noindent {\sc HEIG--VD} \hfill Probabilités et Statistique\newline 
\noindent {Corrigé Etudiant} \hfill 2021-2022\newline
\hrule
\vspace{3mm}
\noindent {\bf Thème : #1} \hfill Solution de la #2
\vspace{5mm}
\hrule
\vspace{7mm}
}
%

%
\newcommand{\separation}{{\begin{center}\rule{10cm}{0.25pt}\end{center}}\noindent} % crée une barre
%

%
% Math
%
\newcommand{\Real}{\mathbb R}
\newcommand{\RPlus}{\Real^{+}}
\newcommand{\norm}[1]{\left\Vert#1\right\Vert}
\newcommand{\abs}[1]{\left\vert#1\right\vert}
\newcommand{\setn}[1]{\left\{#1\right\}_{\scriptscriptstyle n \ge 1}}
\newcommand{\set}[1]{\left\{#1\right\}}
\newcommand{\seq}[1]{\left<#1\right>}
\newcommand{\eps}{\varepsilon}
\newcommand{\To}{\longrightarrow}
\newcommand{\Prob}{\rm{P}}
\newcommand{\F}{\mathcal{F}}
\newcommand{\h}{\mathcal{H}}
\newcommand{\M}{\mathcal{M}}
\newcommand{\N}{\mathcal{N}}
\newcommand{\E}{{\rm E}}
\newcommand{\Hnull}{{\rm H}_{0}}
\newcommand{\Hone}{{\rm H}_{1}}
\newcommand{\Var}{{\rm Var}}
\newcommand{\Cov}{{\rm Cov}}
\newcommand{\sign}{{\rm sign}}
\newcommand{\med}{{\rm med}}
\newcommand{\tr}{{\rm tr}}
\newcommand{\T}{{\text{\tiny \rm T}}}
\newcommand{\minf}{- \, \infty}
\newcommand{\intervalle}[4]{\mathopen{#1}#2\mathpunct{},#3\mathclose{#4}}
\newcommand{\intervalleff}[2]{\intervalle{[}{#1}{#2}{]}}
\newcommand{\intervalleof}[2]{\intervalle{]}{#1}{#2}{]}}
\newcommand{\intervallefo}[2]{\intervalle{[}{#1}{#2}{[}}
\newcommand{\intervalleoo}[2]{\intervalle{]}{#1}{#2}{[}}   

%
% Section
%
\titleformat
{\section} % Section type being modified
[block] % Shape type, can be: hang, block, display, runin, leftmargin, rightmargin, drop, wrap, frame
{\Large\bfseries} % Format of the whole section
{\assignmentQuestionName~\thesection} % Format of the section label
{6pt} % Space between the title and label
{} % Code before the label

\titlespacing{\section}{0pt}{0.5\baselineskip}{0.5\baselineskip} % Spacing around section titles, the order is: left, before and after

%
%	Custom Exercice Environnment
%
\newcommand{\assignmentQuestionName}{Exercice} % The word to be used as a prefix to question numbers

% Environment to be used for each question in the assignment
\newenvironment{exo}{
	\vspace{0.5\baselineskip} % Whitespace before the question
	\section{} % Blank section title (e.g. just Exercice 2)
}

%------------------------------------------------

% Environment for subquestions, takes 1 argument - the name of the section
\newenvironment{subexo}[1]{
	\subsection{#1}
}{
}

%------------------------------------------------

% Command to print a question sentence
\newcommand{\donnee}[1]{
	{Donnée: } 
	\emph{#1}
	\vspace{0.5\baselineskip} % Whitespace afterwards
}

%------------------------------------------------

% Command to print a box that breaks across pages with the question answer
\newcommand{\answer}[1]{#1}
%------------------------------------------------

\frenchspacing




\begin{document}
	\header{probabilités élémentaires}{Série 1}
	%
	% Exercice 1
	%
	\begin{exo}
		\donnee{Un groupe de consommateurs a réalisé une étude pour analyser le service offert par 200 employés de	divers restaurants. On s’intéresse à une possible relation entre la qualité du service et la qualification du personnel (diplômé d’une école hôtelière ou non). Les résultats de l’enquête figurent dans le tableau ci-dessous :}
		\begin{enumerate}[label=\alph*)]
			\item 
			\begin{itemize}
				\item[--] $0.455$;
				\item[--] $0.555$;
				\item[--] $0.305$.
			\end{itemize}
			\item équiprobabilité
		\end{enumerate}
	\end{exo}
	%
	% Exercice 2
	%
	\begin{exo}
		\donnee{fg}
		\begin{multicols}{3}
			\begin{enumerate}[label=\alph*), parsep=0cm, itemsep=3mm, topsep=3mm]
				\item $0.1$
				\item $0.3$
				\item $0.2$
			\end{enumerate}
		\end{multicols}
	\end{exo}
	
	%
	% Exercice 3
	%
	\begin{exo}
		\donnee{df}
		\begin{multicols}{2}
			\begin{enumerate}[label=\alph*), parsep=1mm, itemsep=3mm, topsep=3mm]
				\item\label{q.quatrieme} $\left( \dfrac{5}{6} \right)^{10}$
				\item $1 - \left( \dfrac{5}{6} \right)^{10}$ 
			\end{enumerate}
		\end{multicols}
	\end{exo}
	
	%
	% Exercice 4
	%
	\begin{exo}
		\donnee{df}
		\begin{enumerate}[label=\alph*), parsep=0cm, itemsep=3mm, topsep=3mm]
			\item $P(A) = p$ \hspace{5mm} $P(B) = 2p$ \hspace{5mm} $P(A \cup B) = 0.28$
			\item $p = 0.1$
		\end{enumerate}
	\end{exo}
	
	%
	% Exercice 5
	%
	\begin{exo}
		\donnee{df}
		\begin{enumerate}[label=\alph*), parsep=0cm, itemsep=3mm, topsep=5mm]
			\item $0.49$
			\item $0.23$
		\end{enumerate}
	\end{exo}
	
	%
	% Footer
	%
	\vfill
	\hrule
	\vspace{2mm}
	\noindent {\tiny Corrigé Etudiant - TIC} \hfill {\tt \tiny \today}
\end{document}
