\documentclass[a4paper,11pt]{report}
\usepackage{amsmath}
\usepackage[svgnames]{xcolor}
\usepackage[english,french]{babel}
\usepackage{amsthm}
\usepackage{amssymb}
\usepackage{multicol}
\usepackage{pstricks,pst-node,pst-text,pst-poly,pst-3d}
\usepackage[utf8]{inputenc}
\usepackage{enumitem}
\usepackage{sidenotes}
\setlength{\textwidth}{175mm}
\setlength{\textheight}{255mm}
\setlength{\oddsidemargin}{-10mm}
\setlength{\topmargin}{-15mm}
\setlength{\parskip}{0.2cm}
%
\newcommand{\ds}{\displaystyle}
\newcommand{\scr}{\scriptscriptstyle}
\newcommand{\bs}[1]{\ensuremath{\boldsymbol{#1}}}
\renewcommand{\leq}{\leqslant}
\newenvironment{refer} 
{
	\begin{list}
		{}
		{
			\setlength{\labelwidth}{.5em}
			\setlength{\leftmargin}{0.4cm}
			\setlength{\itemsep}{0cm}
		} 
	}
	{\end{list}}
%\pagenumbering{roman}
%\setcounter{page}{1}
%
\theoremstyle{definition}
\newtheorem{exo}{Exercice}
%
% Math
%
\newcommand{\Real}{\mathbb R}
\newcommand{\RPlus}{\Real^{+}}
\newcommand{\norm}[1]{\left\Vert#1\right\Vert}
\newcommand{\abs}[1]{\left\vert#1\right\vert}
\newcommand{\setn}[1]{\left\{#1\right\}_{\scriptscriptstyle n \ge 1}}
\newcommand{\set}[1]{\left\{#1\right\}}
\newcommand{\seq}[1]{\left<#1\right>}
\newcommand{\eps}{\varepsilon}
\newcommand{\To}{\longrightarrow}
\newcommand{\Prob}{\rm{P}}
\newcommand{\F}{\mathcal{F}}
\newcommand{\h}{\mathcal{H}}
\newcommand{\M}{\mathcal{M}}
\newcommand{\N}{\mathcal{N}}
\newcommand{\E}{{\rm E}}
\newcommand{\Hnull}{{\rm H}_{0}}
\newcommand{\Hone}{{\rm H}_{1}}
\newcommand{\Var}{{\rm Var}}
\newcommand{\Cov}{{\rm Cov}}
\newcommand{\sign}{{\rm sign}}
\newcommand{\med}{{\rm med}}
\newcommand{\tr}{{\rm tr}}
\newcommand{\T}{{\text{\tiny \rm T}}}
\newcommand{\minfty}{- \, \infty}
\def\transf{\quad\smash{\mathop{\longrightarrow}\limits_{}^{S}}\quad} 
\newcommand{\intervalle}[4]{\mathopen{#1}#2\mathpunct{},#3\mathclose{#4}}
\newcommand{\intervalleff}[2]{\intervalle{[}{#1}{#2}{]}}
\newcommand{\intervalleof}[2]{\intervalle{]}{#1}{#2}{]}}
\newcommand{\intervallefo}[2]{\intervalle{[}{#1}{#2}{[}}
\newcommand{\intervalleoo}[2]{\intervalle{]}{#1}{#2}{[}}   
%
\newcommand{\separation}{{\begin{center}\rule{10cm}{0.25pt}\end{center}}\noindent}
%
\frenchspacing
%
\definecolor{vertfonce}{rgb}{0,0.5,0}
%
\begin{document}
%
\thispagestyle{empty}
%
\noindent {\sc HEIG--VD} \hfill Probabilit�s et Statistique\newline 
\noindent {Corrig� Etudiant} \hfill 2021-2022\newline
\hrule
\vspace{3mm}
\noindent {\bf Th�me : probabilit�s �l�mentaires} \hfill Solutions de la s�rie~1
\vspace{5mm}
\hrule
\vspace{7mm}
%
%%%%%%%%%%%%%%%%%%%%%%%%%%%%%%%%%%%%%%%%%%%%%%%%%%%%%%%%%%%%%%%%%%%%%%%%%%%%%%%%%%%%%%%%%%
%
\noindent {\bf Exercice 1}\\[-3mm]
\begin{enumerate}[label=\alph*), parsep=0cm, itemsep=3mm, topsep=3mm]
	\item 
	\begin{itemize}
		\item[--] $0.455$;
		\item[--] $0.555$;
		\item[--] $0.305$.
	\end{itemize}
	\item �quiprobabilit�
\end{enumerate}
\vspace{1mm}
\noindent {\bf Exercice 2}\\[-3mm]
\begin{multicols}{3}
\begin{enumerate}[label=\alph*), parsep=0cm, itemsep=3mm, topsep=3mm]
	\item $0.1$
	\item $0.3$
	\item $0.2$
\end{enumerate}
\end{multicols}
\vspace{1mm}
\noindent {\bf Exercice 3}\\[-3mm]
\begin{multicols}{2}
\begin{enumerate}[label=\alph*), parsep=1mm, itemsep=3mm, topsep=3mm]
	\item\label{q.quatrieme} $\left( \dfrac{5}{6} \right)^{10}$
	\item $1 - \left( \dfrac{5}{6} \right)^{10}$ 
\end{enumerate}
\end{multicols}
\vspace{1mm}
\noindent {\bf Exercice 4}\\[-3mm]
\begin{enumerate}[label=\alph*), parsep=0cm, itemsep=3mm, topsep=3mm]
	\item $P(A) = p$ \hspace{5mm} $P(B) = 2p$ \hspace{5mm} $P(A \cup B) = 0.28$
	\item $p = 0.1$
\end{enumerate}
\vspace{1mm}
\noindent {\bf Exercice 5}\\[-5mm]
\begin{enumerate}[label=\alph*), parsep=0cm, itemsep=3mm, topsep=5mm]
\item $0.49$
\item $0.23$
\end{enumerate}
\separation
%
% 
%%%%%%%%%%%%%%%%%%%%%%%%%%%%%%%%%%%%%%%%%%%%%%%%%%%%%%%%%%%%%%%%%%%%%%%%%%%%%%%%%%%%%%%%%%
%
\vfill
\hrule
\vspace{2mm}
\noindent {\tiny Corrig� Etudiant - TIC} \hfill {\tt \tiny \today}
\end{document}

%\vfill
%$\mbox{}$ \hspace{16cm} ./..
\clearpage
