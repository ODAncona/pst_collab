\begin{exo}
  \donnee{À vue d’oeil, il fait beau sept fois sur dix à Yverdon–les–Bains le jour de la rentrée académique. Votre enseignant de probabilités et statistique dispose de deux sources de prévisions météorologiques indépendantes: le service météorologique suisse qui se trompe deux fois sur cent et une grenouille verte, qui se trompe une fois sur vingt.}
  \begin{subexo}{À l’aide de l’énoncé, donner les probabilités}
    \begin{multicols}{4}
      \begin{enumerate}
        \item	$P(F|E)$
        \item	$P(F|\conj{E})$
        \item	$P(G|E)$
        \item	$P(G|\conj{E})$
      \end{enumerate}
    \end{multicols}
    Pour commencer, modélisons la situation par un arbre avec $G=$ "la grenouille annonce beau" et $F=$ "la météo annonce le beau" et $E=$ "il fait beau".
    \begin{center}
      \begin{tikzpicture}[grow=right]
        \node[] {}
        child {
          node[] {$\conj{E}$}
          child {
            node[end, label=right: {$G$}] {}
            edge from parent
            node[below]  {$\frac{1}{20}$}
          }
          child {
            node[end, label=right: {$F$}] {}
            edge from parent
            node[below]  {$\frac{1}{50}$}
          }
          edge from parent
          node[below]  {$\frac{3}{10}$}
        }
        child {
          node[] {$E$}
          child {
            node[end, label=right: {$G$}] {}
            edge from parent
            node[below]  {$\frac{19}{20}$}
          }
          child {
            node[end, label=right: {$F$}] {}
            edge from parent
            node[below]  {$\frac{49}{50}$}
          }
          edge from parent
          node[below]  {$\frac{7}{10}$}
        };
      \end{tikzpicture}
  \end{center}
  De cet arbre découle:
  \begin{enumerate}
    \item	$P(F|E) = \dfrac{49}{50} = 0,98$
    \item	$P(F|\conj{E}) = 1 - \dfrac{49}{50} = 0,02 $
    \item	$P(G|E) = \dfrac{19}{20} = 0,95$
    \item	$P(G|\conj{E}) = 1 - \dfrac{19}{20} = 0,05$
  \end{enumerate}
  \end{subexo}
  \begin{subexo}{Calculez les probabilités}
    \begin{multicols}{2}
      \begin{enumerate}
        \item $P(\conj{F}\cap G|E)$
        \item $P(\conj{F}\cap G|\conj{E})$
      \end{enumerate}
    \end{multicols}
  \begin{flushleft}
  Nous pouvons distribuer les deux membres de l'inclusion grâce à leur indépendance conditionnelle, $P(\conj{F}\cap G|E) = \dfrac{P(\conj{F}\cap G \cap E) }{P(E)} = P(\conj{F}|E) \cdot P(G|E) $. Nous avons déjà $P(G|E) = \dfrac{19}{20} = 0,95$. Il nous manque $P(\conj{F}|E)$ que l'on peut retrouver à l'aide de $1 - P(F|E) = 0,02$ Ces branches peuvent être détaillées dans l'arbre ci-dessous:
  \end{flushleft}
  \begin{center}
    \tikzstyle{level 1}=[level distance=3.5cm, sibling distance=3.5cm]
    \begin{tikzpicture}[grow=right]
      \node[] {$E$}
      child {
        node[end, label=right: {$\conj{F}$}] {}
        edge from parent
        node[below]  {$\frac{1}{50}$}
      }
      child {
        node[end, label=right: {$F$}] {}
        edge from parent
        node[below]  {$\frac{49}{50}$}
      }
      child {
        node[end, label=right: {$G$}] {}
        edge from parent
        node[above]  {$\frac{19}{20}$}
      }
      child {
        node[end, label=right: {$\conj{G}$}] {}
        edge from parent
        node[above]  {$\frac{1}{20}$}
      };
    \end{tikzpicture}
  \end{center}
  \begin{flushleft}
  Finalement, nous avons $P(\conj{F}\cap G|E) = P(\conj{F}|E) \cdot P(G|E) = \frac{1}{50} \cdot \frac{19}{20} = 0,019$\\
  De manière similaire, $P(\conj{F}\cap G|\conj{E}) = P(\conj{F}|\conj{E}) \cdot P(G|\conj{E}) = \frac{49}{50} \cdot \frac{1}{20} = 0,049$
  \end{flushleft}
  \end{subexo}
\end{exo}
