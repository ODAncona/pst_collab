\begin{exo}
  \donnee{Un signal binaire valant 0 ou 1 est envoyé par câble électrique. La transmission est affectée par des perturbations dites “bruits” : le signal 0 est reçu en 1 avec probabilité 0.2 et le signal 1 est enregistré en 0 avec probabilité 0.1. Définissons les événements suivants:
  \\ $E_0$ : "le signal émis vaut 0"
  \\ $E_1$ : "le signal émis vaut 1"
  \\ $R_0$ : "le signal reçu vaut 0"
  \\ $R_1$ : "le signal reçu vaut 1"
  \\En supposant que le système de transmission émet le signal 0 avec probabilité 0.45, calculer les probabilités de transmission correctes }
\begin{flushleft}
  Modélisons la situation à l'aide d'un arbre nous pouvons déjà remplir les branches $P(R_1 | E_0) = 0,2$ et $P(R_0 | E_1) = 0,1$ ainsi que $P(E_0) = 0,45$ et $P(E_1) = 0,55$. Ensuite, nous pouvons déduire de là $ P(R_0 | E_0) = 1 - 0,2 = 0,8 $ et $P(R_1 | E_1) = 1 - 0,1 = 0,9$.
\end{flushleft}
\begin{center}
  \begin{tikzpicture}[grow=right]
    \node[] {}
    child {
      node[] {$E_0$}
      child {
        node[end, label=right: {$R_0$}] {}
        edge from parent
        node[below]  {$\frac{4}{5}$}
      }
      child {
        node[end, label=right: {$R_1$}] {}
        edge from parent
        node[above]  {$\frac{1}{5}$}
      }
      edge from parent
      node[below]  {$\frac{9}{20}$}
    }
    child {
      node[] {$E_1$}
      child {
        node[end, label=right: {$R_0$}] {}
        edge from parent
        node[below]  {$\frac{1}{10}$}
      }
      child {
        node[end, label=right: {$R_1$}] {}
        edge from parent
        node[above]  {$\frac{9}{10}$}
      }
      edge from parent
      node[above]  {$\frac{11}{20}$}
    };
  \end{tikzpicture}
\end{center}
  \begin{subexo}{$P(E_0 | R_0)$}
    \begin{flushleft}
      Afin de résoudre cette équation, il faut utiliser la formule de Bayes totale:
      \\$P(E_0|R_0) = \dfrac{P(R_0|E_0)\cdot P(E_0)}{P(R_0)}$ avec $P(R_0) = P(R_0 | E_0) \cdot P(E_0) + P(R_0 | E_1)\cdot P(E_1)$
      \\Numériquement:
      \\$P(R_0) = \dfrac{4}{5}\cdot\dfrac{9}{20} +\dfrac{1}{10}\cdot\dfrac{11}{20} = 0,415$
      \\$P(R_0|E_0)\cdot P(E_0) =  \dfrac{4}{5}\cdot \dfrac{9}{20} = \dfrac{9}{25} = 0,36$
      \\ Finalement, $P(E_0|R_0) = \dfrac{0,36}{0,415} = 0,867$
    \end{flushleft}
  \end{subexo}
  \begin{subexo}{$P(E_1 | R_1)$}
    \begin{flushleft}
      De manière similaire:
      \\$P(E_1|R_1) = \dfrac{P(R_1|E_1)\cdot P(E_1)}{P(R_1)}$ avec $P(R_1) = P(R_1 | E_0) \cdot P(E_0) + P(R_1 | E_1)\cdot P(E_1)$
      \\Numériquement:
      \\$P(R_1) = \dfrac{1}{5}\cdot\dfrac{9}{20} +\dfrac{9}{10}\cdot\dfrac{11}{20} = 0,415$
      \\$P(R_1|E_1)\cdot P(E_1) =  \dfrac{9}{10}\cdot \dfrac{11}{20} = 0,495$
      \\ Finalement, $P(E_0|R_0) = \dfrac{0,495}{0,585} = 0,846$
    \end{flushleft}
  \end{subexo}
\end{exo}
