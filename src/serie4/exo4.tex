\begin{exo}
  \donnee{Un serveur de calcul reçoit des requêtes à fréquence régulière.
    Elles sont traitées indépendamment les
    unes aprés les autres dans leur ordre d’arrivée. On suppose que le nombre de
    requêtes par minute qui
    arrivent au serveur peut être modélisé par un processus de Poisson.
    Les requêtes sont envoyées vers le
    serveur à un rythme moyen de 6 requêtes par minute.}

  \begin{subexo}{Déterminer la probabilité qu'en l'espace d'une minute et demi,
      au moins deux reqêtes arrivent au server.}
    Nous sommes dans le cas d'un processus de poisson. Donc
    \[
      P(N(t) = x) = e^{-\lambda t} \cdot \frac{(\lambda t)^x}{x!}
    \]
    Où
    \begin{enumerate}
      \item $N(t)$ compte le nombre d'occurence = 2
      \item $\lambda$ est la fréquence d'occurence de l'événement observé = 6 par minutes
      \item $t$ est le temps durant lequel on compte le nombre d'occurence  = 1.5 minutes
    \end{enumerate}
    Au vu de l'énoncé "au moins" nous savons que nous devons faire le complémentaire puisque nos
    lois de probabilité nous informent toujours sur $P(X < x)$.
    \begin{align*}
      P(N(1.5) \ge 2) = & 1 - \overline{P(N(1.5) \ge 2)} \\
      =                 & 1 - P(N(1.5) < 2)
    \end{align*}
    Comme nous sommes dans un cas discret, nous avons :
    \[
      P(N(1.5) < 2) = P(N(1.5) = 0) + P(N(1.5) = 1)
    \]
    et donc
    \begin{align*}
      P(N(1.5) \ge 2) = & 1 - P(N(1.5) = 0) - P(N(1.5) = 1)                                                                         \\
      =                 & 1 - e^{-6 \cdot 1.5} \cdot \frac{(6 \cdot 1.5)^0}{0!} - e^{-6 \cdot 1.5 } \cdot \frac{(6\cdot 1.5)^1}{1!} \\
      =                 & 1- e^{-6 \cdot 1.5} - 9 \cdot e^{-6 \cdot 1.5}                                                            \\
      =                 & 1- 10e^{-9}
    \end{align*}
  \end{subexo}
  \begin{subexo}{En sachant que les deux requêtes sont arrivées en l'espace d'une minute, calculer
      la probabilité qu'elles soient toutes les deux arrivées dans les 20 premières secondes}

    Nous voyons ici le fameux "en sachant" qui nous dit d'utiliser bayes.
    \[
      P\biggl( N(t_1) = 2 \biggl| N(t_2) = 2\biggr)
    \]
    \begin{enumerate}
      \item $t_1$ = 20 secondes = $\frac{1}{3}$ minutes
      \item $t_2$ = 1 minute
    \end{enumerate}
    Nous mettons tout en minutes pour avoir des unités cohérente par rapport à notre $\lambda$ qui est en minute
    \begin{equation*}
      P\biggl( N(1/3) = 2 \biggl| N(1) = 2\biggr)
      % = & P\biggl( e^{-6 \cdot 1/3} \cdot \frac{(6 \cdot 1/3)^2}{2 !} \biggl|
      % e^{-6 \cdot 1} \cdot \frac{(6 \cdot 1)^2}{2 !}  \biggr)
    \end{equation*}
    Avec
    \begin{align*}
       & P(N(1/3) = 2) = e^{-6 \cdot 1/3} \cdot \frac{(6 \cdot 1/3)^2}{2 !} \\
       & P(N(1) = 2) = e^{-6 \cdot 1} \cdot \frac{(6 \cdot 1)^2}{2 !}
    \end{align*}

    $N(t_1) = 2$ n'est pas compris dans $N(t_2) = 2$ puisque si les requêtes
    arrivent dans les 20 premières secondes, elles sont forcément dans la première minute
    mais la minute n'a pas reçu uniquement 2 requêtes.
    En effet, il peut encore survenir des requêtes durant les 40 secondes suivantes faisant que
    l'évènement $N(t_1) = 2$ s'est réalisé sans que $N(t_2) = 2$ se soit réalisé puisque il y
    aurait eu 3 requêtes ou plus durant la minute avec deux durant les 20 premières secondes.
    Posons A = $(N(1/3) = 2)$ et B = $ (N(1) = 2)$.
    Nous avons $P( A | B)$ = $\frac{P(A \cap B)}{P(B)}$ et
    \[P(A \cap B) = P(N(1/3) = 2) \cdot \underbrace{P(N(2/3)= 0)}_{\textbf{0 requêtes en 40 [s]}}\]
    \begin{equation*}
      P\biggl( N(1/3) = 2 \biggl| N(1) = 2\biggr) 
      =\frac{P(N(1/3) = 2)\cdot P(N(2/3)= 0)}
          {P(N(1) = 2)} 
          \\
      = \frac{ e^{-6 \cdot 1/3} \cdot \frac{(6 \cdot 1/3)^2}{2 !} \cdot e^{-6\cdot 2/3}\cdot \frac{(6*2/3)^0}{0!}}{ e^{-6 \cdot 1} \cdot \frac{(6 \cdot 1)^2}{2 !} }
    \end{equation*}
    Et ainsi
    \[
      P\biggl( N(t_1) = 2 \biggl| N(t_2) = 2\biggr) = \frac{e^{-2}\cdot 2 \cdot e^{-4}}{e^{-6}\cdot 18} = \frac{e^{-4}\cdot 1}{e^{-4}\cdot 9} = \frac{\textbf{1}}{\textbf{9}}
    \]
    Une autre manière de faire est la suivante :
    \[
      P\biggl(N(1/3)=2 \biggl| N(1)=2 \biggr)= \frac{\mathbf{P\bigl( N(1)=2 | N(1/3)=2\bigr)} \cdot P(N(1/3)=2)}{P(N(1)=2)}
    \]
    et la probabilité qu'il y aille deux requêtes en 1 minutes sachant qu'il y en a déjà eu deux en 20 secondes
    c'est egal à la probabilité qu'il n'y en aille 0 en 40 secondes. Donc
    \[ \frac{\mathbf{P(N(2/3)=0)} \cdot P( N(1/3) = 2)}{P(N(1)=2)} = \frac{1}{9}\]
  \end{subexo}
\end{exo}
