\begin{exo}
  \donnee{Un réseau de neurones, outil statistique, est utilisé pour reconnaitre les
    caractères, par exemple les lettres de l'alphabet ou les chiffres arabes, dans un très
    long texte manuscrit. On approche le nombre de caractères interprétés incorrectement
    par le réseau de neurones par une variables aléatoire issue d'une distribution
    de Poisson d'espérance 5}

  \begin{subexo}{Calculer la probabilité qu'au plus 2 caractères d'un très long texte
      donné soient transcrit incorrectement par le réseau de neurones.}
    Avant de commencer la résolution, nous devons noter que \textbf{l'espérance} vaut 5.
    Nous savons que dans ce cas (\textbf{variable de Poisson})
    \[
      \mathbb{E}(X) = \lambda \iff \lambda = 5
    \]
    Nous avons dans le cas d'une variable de Poisson
    \[
      P(X=x) = e^{-\lambda} \cdot \frac{\lambda^x}{x!}
    \]
    Et nous voulons calculer $P(X \le 2)$
    Nous sommes dans un cas \textit{discret} donc nous pouvons
    énumerer les possibilités $\le 2$.
    \begin{align*}
      P(X\le 2) = & P(X=0) + P(X=1) + P(X=2)                                                                                             \\
      =           & e^{-5} \cdot \frac{5^0}{0!} + e^{-5} \cdot \frac{5^1}{1!} + e^{-5} \cdot \frac{5^0}{0!} + e^{5} \cdot \frac{5^2}{2!} \\
      =           & \frac{2}{2} \cdot e^{-5} + \frac{10}{2}  \cdot e^{-5} + \frac{25}{2}\cdot e^{-5}                                     \\
      =           & \frac{37}{2}e^{-5}
    \end{align*}
  \end{subexo}
  \begin{subexo}{En sachant que le réseau de neurones a déjà interprété incorrectement au moins un caractère
      d'un très long texte donné, déterminer la probabilité qu'exactement 2 caractères soient
      transcrits incorrectement}

    \begin{equation}
      P(A|B) = \frac{P(B|A) \cdot P(A)}{P(B)}
    \end{equation}
    Et
    \begin{equation}
      P(A|B) = \frac{P(A \cap B)}{P(B)}
    \end{equation}

    Nous avons des probabilité conditionelles puisqu'il y a le fameux "en sachant".
    Sous forme mathématique, nous avons:

    \[
      P(X=2 | X\ge 1) = \frac{P(X\ge 1 | X=2) \cdot P(X=2)}{P(X\ge 1)}
    \]
    \\

    Cependant, $P(X\ge 1 | X=2)$ vaut 1 et $P(X\ge 1)$ au dénominateur vaut $1-P(X < 1) = 1-P(X=0)$
    Donc nous avons 
    \begin{align}
      P(X=2 | X\ge 1) &= \frac{P(X\ge 1 | X=2) \cdot P(X=2)}{P(X\ge 1)}\\
      &= \frac{P(X=2)}{1- P(X=0)}\\
      &= \frac{e^{-5}\cdot \frac{25}{2}}{1-e^{-5}} = \frac{25\cdot e^{-5}}{2(1-e^{-5})}
      = \frac{25\cdot e^{-5}}{2e^{-5}(\frac{1}{e^{-5}} -1)}\\
      &= \frac{25}{2(e^{5}-1)}
    \end{align}
  \end{subexo}
\end{exo}
