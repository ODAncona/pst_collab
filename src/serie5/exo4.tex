\begin{exo}
  \donnee{Pour modéliser le maximum annuel des débits d'une certaine rivière, les hydrologues utilisent habituellement une variable aléatoire X issue d'une distribution de Gumbel dont la fonction de répartition est donnée par: $F_x(x) = e^{-e^{-\frac{x-a}{b}}}´$ où $x \in \Real$. À partir des maxima annuels des débits relevés de 1971 à 1995 à une station située sur la rivière, les paramètres a et b ont été estimés $25.5$ et $7.98$ respectivement.}
	\begin{subexo}{Calculez la probabilité qu'une année donnée le maximum des débits soit supérieur à $35.5$  $m^3/s$}
  		\begin{center}
  			La fonction de répartition étant déjà donnée avec a et b,
  		\end{center}
		\begin{align*}
			P(X > 35.5) &= 1 - P(X \leq 35.5) \\
			&= 1 - F_x(35.5) \\
	  		&= 1 - e^{-e^{-\frac{x-7.98}{25.5}}}\\
	  		&\approx 0.248
	 	\end{align*}
  	\end{subexo}
  	\begin{subexo}{La période de retour T caractérise la durée moyenne s'écoulant entre deux occurences consécutives d'un même événement. Plus précisément, elle est définie par $ T = \dfrac{1}{1- F_x(x)}$. Calculez le débit maximal annuel x correspondant à une période de retour de 100 ans. Ce débit est interprété comme étant le débit maximal annuel qui sera dépassé en moyenne une fois tous les 100 ans.}
	\begin{center}
		On cherche à calculer T = 100
	\end{center}
	\begin{align*}
		100 &=  \dfrac{1}{1- F_x(x)}\\
		0.99 &= F_x(x) \\
		0.99 &= e^{-e^{-\frac{x-a}{b}}} \\
		-\ln(0.99) &= e^{-\frac{x-a}{b}} \\ 
		\ln{-\ln(0.99)} &= -\frac{x-a}{b} \\ 
		b\cdot \ln{-\ln(0.99)} &= a-x \\ 
		x &= a-b\cdot \ln{-\ln(0.99)} \\ 
		&= 25.5 -7.98\cdot \ln{-\ln(0.99)} \\
		&\approx 62.21
	\end{align*}
	\end{subexo}
\end{exo}
