\begin{exo}
  \donnee{Pour modéliser le maximum annuel des débits d'une certaine rivière, les hydrologues utilisent habituellement une variable aléatoire X issue d'une distribution de Gumbel dont la fonction de répartition est donnée par: $F_x(x) = e^{-e^{-\frac{x-a}{b}}}´$ où $x \in \Real$. À partir des maxima annuels des débits relevés de 1971 à 1995 à une station située sur la rivière, les paramètres a et b ont été estimés $25.5$ et $7.98$ respectivement.}
  	\begin{subexo}{Calculez la probabilité qu'une année donnée le maximum des débits soit supérieur à $35.5$  $m^3/s$}
  	\end{subexo}
  	\begin{subexo}{La période de retour T caractérise la durée moyenne s'écoulant entre deux occurences consécutives d'un même événement. Plus précisément, elle est définie par $ T = \dfrac{1}{1- F_x(x)}$. Calculez le débit maximal annuel x correspondant à une période de retour de 100 ans. Ce débit est interprété comme étant le débit maximal annuel qui sera dépassé en moyenne une fois tous les 100 ans.}
	\end{subexo}
\end{exo}
