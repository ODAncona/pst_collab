\begin{exo}
  \donnee{Le temps en secondes que passe un internaute sur une page d'un site WEB peut être décrit par une variable aléatoire X telle que $ Y = \ln(X)$ est une variable aléatoire issue d'une distribution normale d'espérance $0.5$ et de variance $1$. On dit que X est une variable aléatoire issue d'une distribution log-normale.}
	\begin{subexo}{Exprimer la fonction de répartition $F_x$ de la variable aléatoire $X$ en utilisant la fonction de répartition $\phi$.}
		\begin{center}
			X : "Temps en secondes que passe un internaute sur une page WEB" \\
			Comme $Y = \ln(X)$, on a $Y \sim \N(0.5,1)$
		\end{center}
		\begin{align*}
			F_x(x) &= P(X \leq x) \\
			&= P(\ln(X) \leq \ln(x)) \\
			&= P(Y \leq \ln(x))
		\end{align*}
	Après centrage et réduction,
		\begin{align*}
			F_x(x) &= P(Y \leq \ln(x)) \\
			&= P\left(\frac{Y-\mu}{\sigma} \leq \frac{\ln(x)-\mu}{\sigma}\right) \\
			&= \phi\left(\frac{\ln(x)-\mu}{\sigma}\right)
		\end{align*}
	\end{subexo}
	\begin{subexo}{Calculer la probabilité qu'une page soit regardée pendant plus de 10 secondes.}
		\begin{center}
			Avec $\mu = 0.5$ et $\sigma = 1$,
		\end{center}
		\begin{align*}
			P(X > 10) &= 1- P(X \leq 10) \\
			&= 1- F_x(x) \\
			&= 1 -\phi\left(\frac{\ln(x)-\mu}{\sigma}\right) \\
			&\approx 1 -\phi(1.8) \\
			&\approx 0.036
		\end{align*}
	\end{subexo}
\end{exo}
